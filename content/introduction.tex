% !TeX spellcheck = hu_HU
% !TeX encoding = UTF-8
%----------------------------------------------------------------------------
\chapter{\bevezetes}
%----------------------------------------------------------------------------

A soros végrehajtás alapú érdemi gyorsulás korlátait elértük.
Ha egy modern program a lehető legnagyobb sebességet szeretné elérni, akkor a megfelelő párhuzamos hardver mellett, magának a programnak is parallel algoritmusokat kell alkalmaznia.
Az informatika ezen ágazata már a múlt évezred óta aktív kutatás célja, és sokféle szerteágazó megoldást mutattak be és hoztak létre, mind a kutatók, mind a különböző hardver gyártók.
A dolgozat célja, hogy egy, a közismeretből jobban kiesett, de igazán jól használható lehetőséget, a Linda nyelvet, bemutassa, hogy a mai napig megfelelő párhuzamosítást tud biztosítani egy, a többi alternatívánál egyszerűbb elméleti modell felhasználásával.

Az első fejezet bemutatja az elérhető hardver és szoftver párhuzamosítási lehetőségeket--kitérve azok rövid történelmére--amelyek rendelkezésre állnak egy modern alkalmazás fejlesztése során.
Ezek az eszközök lehetnek egyszerű asztali alkalmazástól kezdve a szuperszámítógépeken futó különféle tudományos számítások.
Szó esik róla, hogy mára, már az asztali gépek is párhuzamos feldolgozást támogató hardverrel készülnek, míg a szupergépek újabb és újabb határokat lépnek át a számítási kapacitásban. %TODO cite Frontier exaflop
A fejezet felvezeti a Linda nyelv történelmi kontextusát is.

A második fejezet egy részletes bemutatást ad a Linda nyelvre és az általa biztosított lehetőségekre.
Megtörténik az egyes nyelvi elemek ismertetése elvi szinten.
A dolgozathoz tartozó megvalósítás is itt kerül tárgyalásra, az általános architektúrától, az egyes elemek megvalósításáig.

A harmadik, és utolsó fejezet a Linda nyelvet tárgyalja tovább, mint egy kiváló oktatási eszközt a párhuzamosítás elveinek bemutatására.
A nyelv előzőekben megismert szintaktikája és szemantikájának egyszerűségére lehetőséget nyújt, hogy az oktatási anyag a tényleges párhuzamosítási elvekre tudjon koncentrálni, anélkül, hogy az egyéb megoldások sajátosságait kellene megtanítani.
Emellett egy gyakorlati példát, egy adott projekt forrásfájljainak párhuzamos fordításának lehetőségét is bemutatja, mint egy olyan gyakorlati probléma, amelyre egy elegáns és egyszerű megoldást tud biztosítani a Linda nyelv.


